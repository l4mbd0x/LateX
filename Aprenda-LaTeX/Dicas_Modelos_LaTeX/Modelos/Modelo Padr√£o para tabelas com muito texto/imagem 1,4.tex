\documentclass[11pt,a4paper]{article}

\usepackage{booktabs}
\usepackage{tabularx}
\usepackage[brazilian]{babel}

\begin{document}

\begin{table}[htb!]
\centering
\caption{Caracteriza\c{c}\~ao das RSSFs segundo a sua configura\c{c}\~ao}
\label{table:economicSchools}   
\begin{tabular}{|l|p{0.2\linewidth}|p{0.6\linewidth}|}
\hline
\multicolumn{3}{|c|}{ \textbf{Configura\c{c}\~ao} }                                                    \tabularnewline \hline
\textbf{Composi\c{c}\~ao} & Homog\^enea   & Rede composta de n\'os que apresentam a mesma capacidade de hardware. Eventualmente os n\'os podem executar softwares diferentes.    \tabularnewline
\cline{2-3}
			  & Heterog\^enea & Rede composta por n\'os com diferentes capacidades de hardware.  \tabularnewline \hline
\textbf{Organiza\c{c}\~ao}& Hier\'arquica & RSSF em que os n\'os est\~ao organizados em grupos (\textit{clusters}). Cada grupo ter\'a um l\'ider (\textit{cluster-head}) que poder\'a ser eleito pelos n\'os comuns. Os grupos podem organizar hierarquias entre si. \tabularnewline 
\cline{2-3}
			  & Plana         & Rede em que os n\'os n\~ao est\~ao organizados em grupos. \tabularnewline \hline
\textbf{Mobilidade}       & Estacion\'aria& Todos os n\'os sensores permanecem no local onde foram depositados durante todo o tempo de vida da rede.  \tabularnewline 
\cline{2-3}
                          & M\'ovel       & Rede em que os n\'os sensores podem ser deslocados do local onde inicialmente foram depositados.     \tabularnewline \hline
                          & Balanceada    & Rede que apresenta uma concentra\c{c}\~ao e distribui\c{c}\~ao de n\'os por unidade de \'area considerada ideal segundo a fun\c{c}\~ao objetivo da rede. \tabularnewline
\cline{2-3}
\textbf{Densidade}	  & Densa         & Rede que apresenta uma alta concentra\c{c}\~ao de n\'os por unidade de \'area. \tabularnewline
\cline{2-3}
			  & Esparsa       & Rede que apresenta uma baixa concentra\c{c}\~ao de n\'os por unidade de \'area. \tabularnewline \hline
\textbf{Distribui\c{c}\~ao}& Irregular    & Rede que apresenta uma distribui\c{c}\~ao n\~ao uniforme dos n\'os na \'area monitorada. \tabularnewline
\cline{2-3}
			  & Regular       & Rede que apresenta uma distribui\c{c}\~ao uniforme de n\'os sobre a \'area monitorada. \tabularnewline \hline
                          & Pequena       & Rede composta de uma centena de elementos de rede. \tabularnewline
\cline{2-3}
\textbf{Tamanho}          & M\'edia       & Rede composta de cem a mil elementos de rede. \tabularnewline
\cline{2-3}
			  & Grande        & Rede composta por milhares de elementos de rede. \tabularnewline \hline

\end{tabular}
\end{table}
\centering
Fonte ....

\end{document}
